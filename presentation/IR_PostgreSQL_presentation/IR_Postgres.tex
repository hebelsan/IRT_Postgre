% Updated by Michael Gertz, April 2017
%%%%%%%%%%%%%%%%%%%%%%%%%%%%

\documentclass{beamer}
%\usepackage[ngerman]{babel}
\usepackage[utf8]{inputenc}

\usepackage{color}
\usepackage{graphicx}
\usepackage{fancybox}

\usepackage{forest}
\usepackage{listings}

\definecolor{dkgreen}{rgb}{0,0.6,0}
\definecolor{mauve}{rgb}{0.58,0,0.82}

\lstdefinestyle{myCustomMatlabStyle}{
	language=C,
	numbers=left,
	stepnumber=1,
	numbersep=10pt,
	keywordstyle=\color{blue},
	commentstyle=\color{dkgreen},
	tabsize=4,
	showspaces=false,
	showstringspaces=false
}

\usepackage{beamerthemesplit}
\usetheme[compress]{Heidelberg}
\definecolor{unirot}{rgb}{0.5976525,0,0}
\usecolortheme[named=unirot]{structure}


\title[IR with PostgreSQL]{Information Retrieval with PostgreSQL}
\author[Alexander Hebel]{Alexander Hebel}
\date{Mai 6, 2020}
\institute[Uni HD]{
Heidelberg University\\
Institute of Computer Science\\
Database Systems Research Group\\
\color{unirot}{vx228@uni-heidelberg.de}}

%---------------------------------------%
%---------- RECURRING OUTLINE ----------%
% have this if you'd like a recurring outline
\AtBeginSection[]  % "Beamer, do the following at the start of every section"
{
\begin{frame}<beamer> 
\frametitle{Outline} % make a frame titled "Outline"
\tableofcontents[currentsection,hideallsubsections]  % show TOC and highlight current section
\end{frame}
}
%----------------------------------------


\begin{document}
\frame[plain]{\titlepage}
\frame{\frametitle{Outline}\tableofcontents[hideallsubsections]}

%========================================
%========================================

\section[Introduction]{Introduction}

\subsection{Introduction}


\frame{
\frametitle{Information Retrieval System (IRS)}

\begin{block}{\centering Information Retrieval (IR)}
	\begin{itemize}
		\item Information retrieval is the science of searching for information in a document		
		\item An IR system is a software system that provides access to books, journals and other documents; stores and manages those documents. 
		\item Web search engines are the most visible IR applications.
	\end{itemize}
\end{block}

\begin{block}{\centering Full-Text-Search}
	The activity of searching through a collection of natural-language documents to locate those that best match a query
\end{block}

}

	
\frame{	
\frametitle{Objectives}

	\begin{block}{}
		To our best knowledge there exists no IRS based on a relational database.
	\end{block}

	\begin{itemize}
		\item Finding different suitable database models
		\item Python api for the database creation and communication
		\item Crawl Wiki pages to gather text data
		\item Store and manage those text documents 
		\item Support search querys containing the boolean operator AND
		\item Create a ranking function for the query results
		\item Compare the ranking of whole Wiki pages and their sections
	\end{itemize}
} % END OF FRAME


\frame{	
	\frametitle{Tsvector and Tsquery}
	
	\begin{itemize}
		\item PostgreSQL provides \textbf{two} data types that are designed to support full text search
		\item The \textbf{tsvector} type represents a document in a form optimized for text search
		\item The \textbf{tsquery} type similarly represents a text query
	\end{itemize}

	\begin{block}{\centering Example Tsvector}
		SELECT to\_tsvector('english', 'The Fat Rats');\\
		\{'fat':2 'rat':3\}
	\end{block}

	\begin{block}{\centering Example Tsquery}
		SELECT to\_tsquery('Fat \& Cats');\\
		\{'fat' \& 'cat'\}
	\end{block}
} % END OF FRAME

\frame{
	\frametitle{Tsvector Pro and Contra}
	
	\begin{columns}[t]
		\begin{column}{.5\textwidth}
			{\color{unirot}Possibilities}
			\begin{itemize}
				\item Full text search
				\item GIN-Index
				\item Automatic tokenization and lemmatization
				\item Adding weights
				\item Predefined ranking function
			\end{itemize}
		\end{column}	
		
		\begin{column}{.5\textwidth}
			{\color{unirot}Limitations} 
			\begin{itemize}
				\item The number of lexemes must be less than $2^{64}$ 
				\item Max position value: 16383
				\item No more than 256 positions per lexeme
				\item Relative small set of manipulation methods
				\item Ranking function must be customized
			\end{itemize}
		\end{column}
	\end{columns}	
} % END OF FRAME

%========================================
%========================================

\section[Realization]{Approach and Realizations}

\subsection{Approach and Realizations}

\iffalse
\frame{
	\frametitle{Realization}
	
	\begin{block}{\centering Wiki crawler}
		\centering
		\begin{itemize}
			\item Based on package wikipedia-api
			\item Takes number of pages and category as input 
			\item Also searches in subcategories
			\item Variable level of subcategories
		\end{itemize}
	\end{block}

	\begin{block}{\centering Database pipeline}
		\centering
		\begin{itemize}
			\item Used package psycopg2 version 2.8.5
			\item Custom converter for tsvector
		\end{itemize}
	\end{block}

	\begin{block}{\centering Document Definition}
		A document in my case is either a page title, seciton title or section text
	\end{block}
} % END OF FRAME
\fi

\frame{	
	\frametitle{Text Data Statistics}
	
	\begin{block}{\centering Document Definition}
		A document in my case is either a page title, seciton title or section text
	\end{block}
	
	\begin{block}{\centering Wikipedia Category "Sports"}
		\centering
		\begin{itemize}
			\item Number of Wiki pages: 2000
			\item Number of sections (captions also count as section): 45756
			\item Total number of lexemes: 1969427
		\end{itemize}
	\end{block}

	\begin{block}{}
		\begin{itemize}
			\item Max number of words of a page: 124136
			\item Max number of lexemes of a page: \textbf{17787}
		\end{itemize}
	\end{block}

	\begin{block}{}
		\begin{itemize}
			\item Max number of words of a section: 24163
			\item Max number of lexemes of a section: \textbf{2785}
		\end{itemize}
	\end{block}
} % END OF FRAME


\iffalse
\frame{	
	\frametitle{Text Data Statistics 2}
	
	\begin{block}{\centering Top 10 Frequent Terms}
		\begin{itemize}
		\item[(1)] "game" \qquad \qquad TF: 14044 \quad DocF: 5370
		\item[(2)] "world" \qquad \qquad TF: 13568 \quad DocF: 6144
		\item[(3)] "championship" $\>$ TF: 12106 \quad DocF: 5475
		\item[(4)] "team" \qquad \quad $\>$ $\>$ TF: 11955 \quad DocF: 4362
		\item[(5)] "sport" \qquad \quad $\>$ $\>$ TF: 10658 \quad DocF: 5014
		\item[(6)] "1" \qquad \quad \qquad $\>$ TF: 10237 \quad DocF: 4029
		\item[(7)] "first" \qquad \qquad $\>$ TF: 10199 \quad DocF: 5073
		\item[(8)] "player" \qquad $\>$ $\>$ $\>$ TF: 9577 $\>$ \quad DocF: 3243
		\item[(9)] "2" \qquad \qquad \quad $\>$ TF: 9090 $\>$ \quad DocF: 3821
		\item[(10)] "play" \quad $\>$ $\>$ $\>$ $\>$ $\>$ $\>$ TF: 9082 $\>$ \quad DocF: 3827
	\end{itemize}
	TF = term frequency \qquad DocF = document frequency
	\end{block}
}
\fi

\iffalse
\frame{
\frametitle{Three Database Model Approaches}

\begin{columns}[t]
	\begin{column}{.33\textwidth}
		{\color{unirot}Tsvector}
		\begin{block}{}
			\centering
			\begin{itemize}
				\item full text search with tsvector
				\item ranking function of tsvector
				\item weighting of tsvector
				\item tokenization and lemmatization
				\item tsquery
			\end{itemize}
		\end{block}	
	\end{column}
	
	\begin{column}{.33\textwidth}
		{\color{unirot}Mix} 
		\begin{itemize}
			\item raw tsvector + word-matrix for each doc
			\item needs more memory
			\item max number tables depends on RAM
			\item customizable
		\end{itemize}
	\end{column}

	\begin{column}{.33\textwidth}
		{\color{unirot}Word-matrix} 
		\begin{itemize}
			\item one big word-matrix
			\item tf easy to retrieve
			\item probably slow
			\item no tokenization and lemmatization
		\end{itemize}
	\end{column}
\end{columns}
\vfill
} % END OF FRAME
\fi

\frame{
\frametitle{Database Model}

\begin{figure}
	\includegraphics[width=1.\textwidth]{db_scheme} 
\end{figure}

} % END OF FRAME


%========================================
%========================================

\section[Customize Tsvectors]{Adding Functionality to Tsvectors}

\subsection{Adding Functionality to Tsvectors}

%----------------------------------------

\def\hilite<#1>{%
  \temporal<#1>{\color{black}}{\color{unirot}}%
               {\color{gray}}}
               
%----------------------------------------


\frame{
	\frametitle{Reason to Extend PostgreSQL}
	
	\begin{block}{}
		\centering
		Only a few manipulation methods for tsvectors exist
	\end{block}

	For example it is not possible to:
	
	\begin{itemize}
		\item[-] count the occurrence of \textbf{all} lexemes in a tsvector \\ (only the number of distinct lexemes)
		\item[-] create a tsvector with an offset for the position index
		\item[-] get the max index of a tsvector
		\item[-] concatenate two tsvectors without changing the position values
	\end{itemize}

	\begin{block}{\centering Solution}
		\centering
		Write your own function either in plpgsql language or in C 
	\end{block}
	
}


\frame{
\frametitle{Custom C-Functions in PostgreSQL}

\begin{columns}[t]
	\begin{column}{.5\textwidth}
		{\color{unirot}Prerequisites}
		\begin{itemize}
			\item Developer version of PostgreSQL
			\item Installation of make
			\item Root privilege on database
		\end{itemize}
	\end{column}
	
	
	\begin{column}{.5\textwidth}
		{\color{unirot}Folder structure} 
		\begin{forest}
			for tree={
				font=\ttfamily,
				grow'=0,
				child anchor=west,
				parent anchor=south,
				anchor=west,
				calign=first,
				edge path={
					\noexpand\path [draw, \forestoption{edge}]
					(!u.south west) +(7.5pt,0) |- node[fill,inner sep=1.25pt] {} (.child anchor)\forestoption{edge label};
				},
				before typesetting nodes={
					if n=1
					{insert before={[,phantom]}}
					{}
				},
				fit=band,
				before computing xy={l=15pt},
			}
			[Extension
			[function.c]
			[Makefile]
			[function.control]
			[function--1.0.sql]
			[README.function]
			]
		\end{forest}
	\end{column}
\end{columns}
\vfill
\begin{block}{\centering Steps}
	\begin{itemize}
		\item[(1)] make install
		\item[(2)] CREATE EXTENSION "extension"
	\end{itemize}
\end{block}
} % END OF FRAME

%========================================
%========================================

\section[Evaluation Objectives]{Evaluation Objectives}

\subsection{Evaluation Objectives}

\frame{
\frametitle{Evaluation Objectives}

\begin{itemize}
	\item Three different database models have been tested
	\item The model mainly using tsvector has proven most suitable
	\begin{itemize}
		\item[+] GIN index for performance
		\item[+] ts\_rank() function for ranking
		\item[+] ts\_query boolean and even phraseto\_tsquery() support
	\end{itemize}
\end{itemize}

\begin{block}{\centering Calculation of Ranking}
	\begin{itemize}
		\item[-] Function ts\_rank() of PostgreSQL
		\item[-] Considers word occurrences and distance between words
	\end{itemize}
\end{block}

\begin{block}{\centering Relationship Between Page and Section Ranking}
	\begin{itemize}
		\item \textbf{section ranking}: ranking / num\_words\_of\_section
		\item \textbf{page ranking}: sum\_of\_rankings / num\_words\_of\_page
	\end{itemize}
\end{block}

}

\frame{
	\frametitle{Query:"game", Sorted by Max Section Ranking}
	
	\begin{figure}
		\includegraphics[width=1.\textwidth]{ordered_by_rmax_together} 
	\end{figure}
}

%\frame{
%	\frametitle{Query:"game AND team AND ball", Sorted by Max Section Ranking}
	
%	\begin{figure}
	%	\includegraphics[width=.95\textwidth]{tripel_ordered_by_rmax_together} 
%	\end{figure}
%}

\frame{
	\frametitle{Distance Between Rankings}
	
	\begin{block}{\centering Calculating Distance}
		\begin{itemize}
			\item Two sorted lists (section\_rankings and page\_rankings)
			\item Distance from first section in section\_rankings to position of page in page\_rankings containing this section and so on...
		\end{itemize}
	\end{block}
	
	\begin{columns}[t]
		\begin{column}{.5\textwidth}
			{\color{unirot}Query "game"}
			\begin{itemize}
				\item result pages: 1176
				\item Top 10 - 289.5 avg(dist) 
				\item Top 20 - 250.9 avg(dist) 
				\item Top 30 - 183.2 avg(dist) 
				\item Top 40 - 155.5 avg(dist) 
			\end{itemize}
		\end{column}
		
		
		\begin{column}{.5\textwidth}
			{\color{unirot}Query "game \& team \& ball"} 
			\begin{itemize}
				\item result pages: 274
				\item Top 10 - 36.0 avg(dist)
				\item Top 20 - 40.6 avg(dist)
				\item Top 30 - 36.6 avg(dist)
				\item Top 40 - 34.0 avg(dist)
			\end{itemize}
		\end{column}
	\end{columns}
	
}

%========================================
%========================================

\section[Conclusion]{Conclusion}

\subsection{Conclusion}

\frame{
	\frametitle{Conclusion and Future Work}
	\begin{block}{\centering Conclusion}
		\begin{itemize}
			\item An IR system made of a relational database is possible
			\item Rankings for sections and page return total different results
			\item Tsvector has a lot of potential
			\item PostgreSQL is easy customizable
			\item Solr gives a little bit different ranking results
		\end{itemize}
	\end{block}

	\begin{block}{\centering Future Work}
		\begin{itemize}
			\item Improve the ranking algorithm with tf idf information (ts\_stat)
			\item Boost certain terms of a query
			\item Proximity query (aka sloppy phrase query)
			\item Tests on big datasets
		\end{itemize}
	\end{block}	
}


\frame{\frametitle{Questions}
	%\begin{figure}
	%\includegraphics[width=.8\textwidth]{questions} 
	%\end{figure}
	\vspace*{2.3cm}\begin{center}\begin{LARGE}\textbf{Questions}\end{LARGE}\end{center}
	\vspace*{2cm}
}

\frame{
	\frametitle{Query:"game", Sorted by Sum of Section Rankings}
	
	\begin{figure}
		\includegraphics[width=1.\textwidth]{2000_only_sec_sorted_by_sum} 
	\end{figure}
}

\end{document}
